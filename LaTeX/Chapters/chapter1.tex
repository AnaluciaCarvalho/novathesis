%%%%%%%%%%%%%%%%%%%%%%%%%%%%%%%%%%%%%%%%%%%%%%%%%%%%%%%%%%%%%%%%%%%%
%% chapter1.tex
%% UNL thesis document file
%%
%% Chapter with introduciton
%%%%%%%%%%%%%%%%%%%%%%%%%%%%%%%%%%%%%%%%%%%%%%%%%%%%%%%%%%%%%%%%%%%%
\chapter{Introduction}
\label{cha:introduction}

This work is licensed under the Creative Commons Attribution-NonCommercial 4.0 International License.
To view a copy of this license, visit \url{http://creativecommons.org/licenses/by-nc/4.0/}.


\todo[inline]{A a note in a line by itself.}

This is the first occurrence of an abbreviation: \gls{abbrev}.

And now the second occurrence of the same abbreviation: \gls{abbrev}.

And a new acronym with capital letter: \Gls{xpt} and reused \gls{xpt}.

Lets add the term ``\gls{computer}'' to the glossary!

Please note that 
\begin{center}
	\textbf{\large this package and template are not official for FCT/UNL}.
\end{center}
